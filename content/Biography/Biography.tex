%\begin{description}

\vspace{2ex}\noindent\textbf{M. Belkacemi} received M.Sc. in Signal and Image processing delivered by the Pierre et Marie CURIE University (UPMC). 
He is currently carrying out a Ph.D. at the University of Burgundy. His research interests include active thermography for Non-Destructive Testing (NDT) and non conventional imaging systems for 3D digitization.

\vspace{2ex}\noindent\textbf{C. Stolz} received his M.Sc. degree in automatics and industrial computing system in 1995 from the Universit� de Haute Alsace, Mulhouse, France. He obtained his PhD in optical signal processing from the same university in 2000. In 2001, he was appointed to his position of assistant professor in the Computing, Electronic, Imaging Laboratory (LE2I) of the Universite de Bourgogne. His research interests mainly concern optical and digital image processing, more particularly polarimetric methods applied to shape measurement.

\vspace{2ex}\noindent\textbf{A. Mathieu} obtained a PhD from University of Burgundy, in 2005. Since 2008, he is lecturer at University of Burgundy. He is a member from laboratory �Interdisciplinaire Carnot de Bourgogne� (ICB UMR 6303 CNRS). His research covers fields such as thermal and mechanical engineering of assembly obtained by welding process, i.e. Laser, arc . In the work presented here, Alexandre carried out the thermal calculations with COMSOL software. 

\vspace{2ex}\noindent\textbf{G. Lemaitre} received B.Sc. degree (Hons) in electrical, signal, and image engineering from the Universte de Bourgogne and the M.Sc. Erasmus Mundus Master of Excellence (Hons) in Vision and Robotics co-jointly delivered by the Universite de Bourgogne, Universitat de Girona, and Heriot-Watt University. 
He is currently carrying out a co-joint Ph.D. at the Universite de Bourgogne and the Universitat de Girona. His research interests include machine learning applied to computer vision and medical imaging.

\vspace{2ex}\noindent\textbf{J. Massich}  is a post-doctoral researcher at uB, France, in Le2i -Laboratoire Electronique, Informatique et Image (UMR CNRS 6306).
He recieved the degree in Computer Science in 2007, the Erasmus Mundus Master Course on Computer Vision and Robotics (ViBOT) in 2009, and the PhD in 2013.
He worked as a researcher at University of Girona (UdG), Texas Tech University (TTU), and University of Bourgundy (uB).

\vspace{2ex}\noindent\textbf{O. Aubreton} received the aggregation examination in 2000 and the D.E.A. degree (equivalent to the M.S. degree) in image processing in 2001.
Since September 2005, he has been an Assistant Professor with the Laboratory Le2i (Vision
3-D team), Institut Universitaire de Technologie,Le Creusot, France. His research interests include the design, development implementation, and testing of silicon retinas for pattern matching and pattern recognition.


%\end{description}